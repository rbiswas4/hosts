\documentclass{article}
\input{preamble}
\begin{document}
\section{Objectives}
Matching supernovae to their hosts is important for (a) obtaining redshifts of 
supernovae via spectroscopic redshifts of hosts (in cases where spectrosscopic
followup of supernovae themeselves is not feasible) as well as (b) to obtain 
properties of the host that may condition the SN properties through 
environmental dependence.


\section{2D Example}
To warm up, imagine that galaxies are always disks and `face on`. In reality, we know this may be potentially better modelled through a triaxial or bulge and disk setup, with random orientiations in the sky. The 2D simulation/analysis probably misses issues related to projections, but we will not address that now.
\begin{enumerate}
\item Assume that the surface brightness is proportional to the surface density of stars in the galaxy, which are related to teh probability of having a SN. We are then assuming that the SN surface density in a galaxy is given by 
    \be
    \sigma(r,\theta) = l (r, \theta) 
    \ee
    where $l(r, \theta)$ is surface brightness density of the galaxy.
\item The surface brightness density is related to the observed angular size of the galaxy (unless the galaxy is not resolved, and things are entirely determined by the PSF?) 
\item This method can be used if we extract the physical profile of the galaxy 
    from the data. Requires a fitting code. 
\end{enumerate}
\section{Questions}
\begin{enumerate}
    \item Do the distribution of physical $a, b$ which are the semi-major and semi-minor axes converted to physical units respectively, become narrow at high redshift? I am wondering if these quantities become entirely determined by the psf when the galaxy is not well-resolved? I have no idea.
\end{enumerate}
\end{document}
